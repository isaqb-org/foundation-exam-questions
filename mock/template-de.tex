\documentclass[12pt,a4paper]{article}

\usepackage{a4}
\usepackage[utf8]{inputenc}
\usepackage{ngerman}
\usepackage{amssymb}
\usepackage{tabularx}
\usepackage{calc}
\usepackage{xparse}
\usepackage{textcomp}

\newcounter{examQuestionCount}
\setcounter{examQuestionCount}{
  %EXAMQUESTIONCOUNT%
}  
\newcounter{examTotalPoints}
\setcounter{examTotalPoints}{
  %EXAMTOTALPOINTS%
}
\newcounter{examMinPoints}
\setcounter{examMinPoints}{\theexamTotalPoints * 60 / 100}

\newif\ifsolutions

% \solutionstrue or nothing
{{solutions}}

\newcommand{\examNo}{$\square$}
\newcommand{\examYes}{\ifsolutions{$\boxtimes$}\else{$\square$}\fi}

\newcounter{examQuestion}\setcounter{examQuestion}{1}

% https://tex.stackexchange.com/questions/17036/why-cant-the-end-code-of-an-environment-contain-an-argument
\NewDocumentEnvironment{examQuestionStem}{mmm}%
  {% avoid mustache
  {\small\arabic{examQuestion}\stepcounter{examQuestion} --- \textit{Id: #1}}\par%
    \begin{minipage}[t]{0.85\textwidth}}%
  {\end{minipage} \hfill\mbox{\small [#2 #3]}\par\medskip}

% \newcommand{\examQuestionStem}[4]{\textbf{#2}\hfill \mbox{[#3 #4]}\par}
\newcommand{\examQuestionInstructions}[1]{\textit{#1}\par}

\newenvironment{examAQuestion}{\begin{itemize}}{\end{itemize}}
\newcommand{\examQuestionAItem}[3]{\item[#3] #1) #2}

\newenvironment{examPQuestion}{\begin{itemize}}{\end{itemize}}
\newcommand{\examQuestionPItem}[3]{\item[#3] #1) #2}

% https://tex.stackexchange.com/questions/573100/defining-a-new-environment-that-contains-tabularx-inside-another-environment-e
\newenvironment{examKQuestion}%
  {\flushleft\tabularx{\textwidth}{cccX}}%
  {\endtabularx\endflushleft}
\newcommand{\examQuestionKHeader}[2]{#1 & #2\\}
\newcommand{\examQuestionKItem}[4]{#3 & #4 & #1) & #2\\}

\setlength{\parskip}{3pt}
\setlength{\parindent}{0pt}

\newcommand{\registered}{$^\textrm{\tiny\textregistered}$}

\begin{document}

\begin{titlepage}
  % FIXME Logo
  \begin{center}
   {\huge\bfseries Beispielprüfung

     iSAQB\registered{} Certified Professional for\\ Software Architecture ---\\[2ex]
     
     Foundation Level (CPSA-F\registered{})}

  \bigskip
  
  {\large International Software Architecture Qualification Board
    e. V.}

  \bigskip

  Document version: {{documentVersion}}\\
  Based on curriculum - version {{curriculumVersion}}
\end{center}

\vspace*{\fill}

  % FIXME: logo
\end{titlepage}

\section*{Prüfungsregeln}

Die vorliegende Prüfung ist eine Beispielprüfung, welche in Form und
Umfang an die Zertifizierungsprüfung des Certified Professional for
Software Architecture ~-- Foundation Level (CPSA-F\registered{})
angelehnt ist. Sie dient der Veranschaulichung der echten
iSAQB\registered{}-CPSA\registered{}--Prüfung sowie der entsprechenden
Prüfungsvorbereitung.

Die Beispielprüfung besteht aus \theexamQuestionCount{} Multiple-Choice-Fragen, welche je
nach Schwierigkeitsgrad mit 1 bis 2 Punkten bewertet werden können. Es
müssen zum Bestehen der Prüfung mindestens 60 Prozent erreicht
werden. In dieser Probeprüfung können \theexamTotalPoints{} Punkte erreicht werden, zum
Bestehen wären \theexamMinPoints{} Punkte erforderlich.

Grundsätzlich gelten folgende Hinweise: Richtige Antworten ergeben
Pluspunkte, falsche Antworten führen zu Punktabzug, jedoch nur in
Bezug auf die jeweilige Frage. Führt die falsche Beantwortung einer
Frage zu einem negativen Punktergebnis, so wird diese Frage mit
insgesamt 0 Punkten bewertet.  Falls Sie mehr Kreuze setzen als
gefordert, erhalten Sie grundsätzlich null Punkte.

Es gibt in dieser Beispielprüfung (wie auch in der Originalprüfung)
nur folgende drei Typen von Prüfungsfragen:

\begin{description}
\item[A-Fragen (Einfachauswahlfragen, "`Auswahl"')] Wählen Sie zu
  einer Frage aus der Liste von Antwortmöglichkeiten die einzig
  korrekte Antwort aus. Es gibt nur eine korrekte Antwort. Sie
  erhalten die angegebene Punktzahl für das Ankreuzen der korrekten
  Antwort.
\item[P-Fragen (Mehrfachauswahlfragen, "`Pick"')] Wählen Sie zu einer
  Frage aus der Liste von Antwortmöglichkeiten die im Text vorgegebene
  Anzahl von zutreffenden oder korrekten Antworten aus. Kreuzen Sie
  maximal so viele Antworten an, wie im Einleitungstext verlangt
  werden. Sie erhalten für jede korrekte Antwort anteilig 1/n der
  Gesamtpunkte. Für jedes nicht-korrekte Kreuz wird 1/n der Punkte
  abgezogen.
\item[K-Fragen (Klärungsfragen, "`Kreuz"')] Wählen Sie zu einer Frage
  die korrekte der beiden Optionen zu jeder Antwortmöglichkeit aus
  ("`richtig"' oder "`falsch"' bzw. "`zutreffend"' oder "`nicht
  zutreffend"'). Sie erhalten für jedes korrekt gesetzte Kreuz anteilig
  1/n der Punkte. Nicht korrekt gesetzte Kreuze führen zum Abzug von
  1/n der Punkte. Wird in einer Zeile KEINE Antwort ausgewählt, so
  gibt es weder Punkte noch Abzüge.
\end{description}
%
Zur genaueren Erläuterung der Fragetypen und Punkteverteilung stehen
weitere Informationen unter der Prüfungsregeln des CPSA-F zur
Verfügung.

Die Bearbeitungsdauer beträgt 75 Minuten für Muttersprachler und 90
Minuten für Nicht- Muttersprachler. Um eine möglichst authentische
Prüfungsvorbereitung zu gewährleisten, sollte die Bearbeitungszeit
eingehalten sowie auf jegliche Hilfsmittel (wie Seminarunterlagen,
Bücher, Internet etc.) verzichtet werden.

Im Anschluss erfolgt die Auswertung der Prüfung mit Hilfe der
Musterlösung. Sofern der iSAQB\registered{} e.V. als Quelle und
Copyright-Inhaber angegeben wird, darf die vorliegende Beispielprüfung
im Rahmen von Schulungen eingesetzt, zur Prüfungsvorbereitung genutzt
oder unentgeltlich weitergegeben werden.

Es ist ausdrücklich untersagt, diese Prüfungsfragen in einer echten
Prüfung zu verwenden.

\newpage

\setcounter{examQuestion}{1}

%EXAMQUESTIONS%

\end{document}
