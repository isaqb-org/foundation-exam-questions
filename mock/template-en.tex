\documentclass[12pt,a4paper]{article}

\usepackage{a4}
\usepackage[utf8]{inputenc}
\usepackage{amssymb}
\usepackage{tabularx}
\usepackage{calc}
\usepackage{xparse}
\usepackage{textcomp}

\newcounter{examQuestionCount}
\setcounter{examQuestionCount}{
  %EXAMQUESTIONCOUNT%
}  
\newcounter{examTotalPoints}
\setcounter{examTotalPoints}{
  %EXAMTOTALPOINTS%
}
\newcounter{examMinPoints}
\setcounter{examMinPoints}{\theexamTotalPoints * 60 / 100}

\newif\ifsolutions

% \solutionstrue

\newcommand{\examNo}{$\square$}
\newcommand{\examYes}{\ifsolutions{$\boxtimes$}\else{$\square$}\fi}

\newcounter{examQuestion}\setcounter{examQuestion}{1}

% https://tex.stackexchange.com/questions/17036/why-cant-the-end-code-of-an-environment-contain-an-argument
\NewDocumentEnvironment{examQuestionStem}{mmm}%
  {{\small\arabic{examQuestion}\stepcounter{examQuestion} --- \textit{Id: #1}}\par%
    \begin{minipage}[t]{0.85\textwidth}}%
  {\end{minipage} \hfill\mbox{\small [#2 #3]}\par\medskip}

% \newcommand{\examQuestionStem}[4]{\textbf{#2}\hfill \mbox{[#3 #4]}\par}
\newcommand{\examQuestionInstructions}[1]{\textit{#1}\par}

\newenvironment{examAQuestion}{\begin{itemize}}{\end{itemize}}
\newcommand{\examQuestionAItem}[3]{\item[#3] #1) #2}

\newenvironment{examPQuestion}{\begin{itemize}}{\end{itemize}}
\newcommand{\examQuestionPItem}[3]{\item[#3] #1) #2}

% https://tex.stackexchange.com/questions/573100/defining-a-new-environment-that-contains-tabularx-inside-another-environment-e
\newenvironment{examKQuestion}%
  {\flushleft\tabularx{\textwidth}{cccX}}%
  {\endtabularx\endflushleft}
\newcommand{\examQuestionKHeader}[2]{#1 & #2\\}
\newcommand{\examQuestionKItem}[4]{#3 & #4 & #1) & #2\\}

\setlength{\parskip}{3pt}
\setlength{\parindent}{0pt}

\newcommand{\registered}{$^\textrm{\tiny\textregistered}$}

\begin{document}

\begin{titlepage}
  % FIXME Logo
  \begin{center}
   {\huge\bfseries Mock Exam

     iSAQB\registered{} Certified Professional for\\ Software Architecture ---\\[2ex]
     
     Foundation Level (CPSA-F\registered{})}

  \bigskip
  
  {\large International Software Architecture Qualification Board
    e. V.}

  \bigskip

  % FIXME: mustache?
  Document version: 2020.1-EN-rev5\\
  Based on curriculum - version V5.1-EN; January 2, 2020

\end{center}

\vspace*{\fill}

  % FIXME: logo
\end{titlepage}

\section*{Examination Rules}

This examination is a mock exam, which is based on the certification
exam of the Certified Professional for Software Architecture ~--
Foundation Level (CPSA-F\registered{}) in form and scope. It serves to illustrate
the real iSAQB\registered{} CPSA\registered{} examination as well as to prepare for the
corresponding exam.

The mock exam consists of \theexamQuestionCount{} multiple-choice
questions, which can be evaluated with 1 or 2 points depending on the
level of difficulty. At least 60 percent must be achieved to pass the
exam.

\theexamTotalPoints{} points can be achieved in this mock examination,
you would need \theexamMinPoints{} points to pass.

The following general rules apply: Correct answers result in plus
points, incorrect answers result in a deduction of points, but only
with regard to the respective question. If the wrong answer to a
question leads to a negative score, this question is evaluated with a
total of 0 points.

The multiple-choice questions of the mock exam are divided into three
types of questions:

\begin{description}
\item[A-Questions (Single Choice, Single Correct Answer)] Select the
  only correct answer to a question from the list of possible
  answers. There is only one correct answer. You receive the specified
  score for selecting the correct answer. Depending on the level of
  difficulty, you can achieve a score of 1 or 2 points.
\item[P-Questions (Pick-from-many, Pick Multiple)] Select the number
  of correct answers given in the text from the list of possible
  answers to a question. Select just as many answers as are required
  in the introductory text. You receive 1/n of the total points for
  each correct answer. For each incorrect cross, 1/n of the points are
  deducted. The score is 1 or 2 points depending on the level of
  difficulty.
\item[K-Questions (Allocation Questions, Choose Category)] For a
  question, select the correct of the two options for each answer
  choice ("correct" or "incorrect" or "applicable" or "not
  applicable"). You will receive 1/n of the points for each correctly
  placed cross. Incorrectly placed crosses result in the deduction of
  1/n of the points. If NO answer is selected in a line, there are
  neither points nor deductions. The score is 1 or 2 points depending
  on the level of difficulty.
\end{description}

For a more detailed explanation of the question types and scoring
system, further information is available in the CPSA-F examination
guide.

The processing time is 75 minutes for native speakers and 90 minutes
for non-native speakers. In order to ensure that the preparation for
the exam is as authentic as possible, the processing time should be
adhered to and any aids (such as seminar materials, books, internet,
etc.) should not be used.

The exam can subsequently be evaluated using the solution for this
mock exam.

Given that the iSAQB\registered{} e.V. is indicated as source and copyright
holder, the present mock exam may be used in the context of training
courses, for exam preparation or it may be passed on free of charge.

However, it is explicitly prohibited to use these exam questions in a
real examination.

\newpage

\setcounter{examQuestion}{1}

%EXAMQUESTIONS%

\end{document}
